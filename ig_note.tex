\documentclass{article}
\usepackage[autonum]{tchdr}
\usepackage{amsmath, amsfonts, graphicx, float, color, tikz, listings, amsthm}
\usetikzlibrary{bayesnet}
\title{Note of Information Geometry Method}
\newcommand{\tre}{\cdots}
\newcommand{\Qexp}{\mcQ^\text{exp}}
\newcommand{\Qfull}{\mcQ^\text{full}}
\newcommand{\kgibbs}{K^\text{Gibbs}}
\newcommand{\kbgibbs}{K^\text{blockGibbs}}
\newcommand{\ytest}{y_\text{test}}
\newcommand{\ytrain}{y_\text{train}}
\newcommand{\xtest}{x_\text{test}}
\newcommand{\xtrain}{x_\text{train}}
\newcommand{\todo}[1]{\text{\textcolor{red}{#1}}}
\makeatletter
\newcommand*\dashline{\rotatebox[origin=c]{90}{$\dabar@\dabar@\dabar@$}}
\author{Peiyuan Zhu}
\begin{document}
	\maketitle
	\section{How to read}
	\begin{enumerate}
		\item Based on book 'Methods of Information Geometry' by Arami
		\item If a reference is not immediately making sense, then it's an indication that the reference isn't completely understood. The next goal then is to understand that reference
		\item The theory can be accompained by learning specific examples.
		\item \url{http://www.robots.ox.ac.uk/~lsgs/posts/2019-09-27-info-geom.html}
		\item \url{https://bjlkeng.github.io/posts/tensors-tensors-tensors/}
		\item \url{https://wiseodd.github.io/techblog/2019/02/22/riemannian-geometry/}
		\item \url{https://people.math.ethz.ch/~salamon/PREPRINTS/diffgeo.pdf}
		\item Typing is very time-consuming and in general unnecessary. Instead, I would listen to the mac assistant reading the passages to me while I try to figure out every step of the derivation. 
	\end{enumerate}
	\section{Differentiable manifolds}
	\begin{enumerate}
		\item $S,Q$ manifold
		\item $n,m$ dimension of manifold
		\item $\psi,\varphi$ chart. ; $S=\{p_\xi=p(x;\xi)|\xi=[\xi^1,\cdots,\xi^n]\in\Xi\}$. For example
		\begin{enumerate}
			\item $S=\reals^n$
			\item $S=S^1,\phi(p)=\frac{x}{1-y},\varphi(p)=\frac{x}{1+y}$
			\item $S=S^n,\phi(p)=\frac{x}{1-z},\varphi(p)=\frac{x}{1+z}$
			\item $S=\left\{\frac{1}{\sqrt{2\pi}\sigma}\exp\left(-\frac{(x-\mu)^2}{2\sigma^2}\right),[\mu,\sigma]\in\reals\times\preals\right\}$
			\item $S=\left\{p_\xi=p(x;\xi)|\xi=[\xi^1,\cdots,\xi^n]\in\Xi\right\}$
		\end{enumerate}
		\item Coordinate transformation from $\psi$ to $\varphi$ \[\psi\circ\varphi^{-1}\qquad(1.1)\] For example 
		\item Motivation: To consider $S$ as a manifold means that one is interested in investigating those properties of $S$ which are invariant under coordinate transformations. In particular, differential geometry analyzes the geometry of objects using differ- ential operators with respect to a variety of functions on $S$, and it would be problematic if these operators depended fundamentally on the choice of coordi- nates. Hence it is necessary to restrict the coordinate systems to those which allow smooth transformations between each other.
		\item $C^\infty$ sufficiently smooth
		\item $C^\infty$-differmorphism: \[\psi\circ\varphi^{-1}\cr\psi^{-1}\circ\varphi\]are both $C^\infty$
		\item $\xi^i,\rho^i$ i-th coordinate map of $\psi,\varphi$\[\xi^i(p)=\xi^i\in\reals\cr\rho^i(p)=\rho^i\in\reals\]
		\item $\xi,\rho$ coordinate map of $\psi,\varphi$
		\[\phi=[\xi^i]\in\reals^n\cr\psi=[\rho^i]\in\reals^n\]
		\item Surrogate of $f$ \[\barf\defined f\circ\varphi^{-1}\]
		\item Partial derivative of $f$ \[\D{f}{\xi^i}\defined\D{\barf}{\xi^i}\circ\varphi\] 
		\item Higher order partial derivative of $f$ \[\frac{\partial^2f}{\partial\xi^j\partial\xi^i}=\frac{\partial}{\partial\xi^j}\frac{\partial f}{\partial\xi^i}\]
		\item $\mcF(S)=\{f:S\rightarrow\reals:f\in C^\infty(S)\}$ an algebra of differentiable functions
		\item  \[\sum_{j=1}^n\D{\xi^i}{\rho^j}\D{\rho^j}{\xi^k}=\sum_{j=1}^n\D{\rho^i}{\xi^j}\D{\xi^j}{\rho^k}=\delta_k^i\qquad(1.2)\] 
		or in Einstein summation \[\D{\xi^i}{\rho^j}\D{\rho^j}{\xi^k}=\D{\rho^i}{\xi^j}\D{\xi^j}{\rho^k}=\delta_k^i\cr\]
		\item  For $f\in\mcF$: \[\D{f}{\rho^j}&=\sum_{i=1}^n\D{\xi^i}{\rho^j}\D{f}{\xi^i}\cr\D{f}{\xi^j}&=\sum_{i=1}^n\D{\rho^i}{\xi^j}\D{f}{\rho^i}\qquad(1.3)\]
		or in Einstein summation
		\[\D{f}{\rho^j}&=\D{\xi^i}{\rho^j}\D{f}{\xi^i}\cr\D{f}{\xi^j}&=\D{\rho^i}{\xi^j}\D{f}{\rho^i}\]
	\end{enumerate}
	\section{Tangent vectors and tangent spaces}
	\begin{enumerate}
		\item $\gamma:I\rightarrow S$ a function
		\item $\gamma^i(t)\defined\xi^i(\gamma(t))$
		\item Surrogate of $\gamma$: $\bar{\gamma}\defined[\gamma^1(t),\cdots,\gamma^n(t)]$
		\item $\gamma$ is a $C^\infty$ curve if $\bar{\gamma}$ is $C^\infty$ 
		\item Motivation: if we define tangent vector at $p$ such that $\gamma(a)=p$ as:
		\[\left(\der{\gamma}{t}\right)_p\equivalentto\dotgamma(a)=\lim_{h\rightarrow0}\frac{\gamma(a+h)-\gamma(a)}{h}\qquad(1.4)\]
		the `subtraction' in the numerator may not be meaningful on manifold that is not a real line. 
		\item Idea: But we can differentiate any real-valued function on the manifold along curve $\gamma$ by $\der{}{t}f(\gamma(t))$. 
		\item For $f\in\mcF$ we have $f(\gamma(t))=\barf(\bar{\gamma}(t))=\barf(\gamma^1(t),\cdots,\gamma^n(t)))$ in coordinate $\xi$. So by chain rule: 
		\[\der{}{t}f(\gamma(t))=\left(\D{\barf}{\xi^i}\right)_{\bar{\gamma}(t)}\der{\gamma^i(t)}{t}=\left(\D{f}{\xi^i}\right)_{\gamma(t)}\der{\gamma^i(t)}{t}\qquad(1.5)\]
		\item Define tangent vector at $p$ as the directional derivative operator along $\gamma$:
		\[\left(\der{\gamma}{t}\right)_p\equivalentto\dotgamma(a):f\mapsto\der{}{t}f(\gamma(t))|_{t=a}\] 
		\item By evaluating $\gamma$ at $a$ (1.5) becomes:
		\[
			\left(\der{\gamma}{t}\right)_p=\dotgamma(a)=\dotgamma^i(a)\left(\D{}{\xi^i}\right)_p\qquad(1.6)
		\]
		where notation $\dotgamma(a)=\der{}{t}\gamma^i(t)|_{t=a}$ is used to simplify the equation and is a rescaling of the operator $\left(\D{}{\xi^i}\right)_p:f\mapsto\left(\D{f}{\xi^i}\right)_p$.
		\item Therefore $\left(\D{f}{\xi^i}\right)_p$ is also the basis tangent vector $e_i$ that goes through point $p$ and is parallel to the $i$-th coordinate curve
		\item When the manifoldis the real (1.6) is equivalent to to (1.4)
		\item Tangent space $T_p(S)=\left\{c_i\left(\D{}{\xi^i}\right)_p|[c^1,\cdots,c^n]\in\reals^n\right\}$. For example
		\begin{enumerate}
			\item Tangent space of graph of a differentiable function $f:\reals\rightarrow\reals$, $S=\graph f$ at $p=(x,y)$ \[T_p(S)=\left\{\left(p,\lambda\cdot\begin{pmatrix}1\cr f'(x)\end{pmatrix}\right):\lambda\in\reals\right\}\]
			\item Tangent space of the sphere $T_p(S^n)=\left\{p'\in\reals^{n+1}|\langle p',p\rangle=0\right\}$
			\item Tangent space of the quadratic form $S=\{p\in\reals^k|p^TAp=c\}$ \[T_p(S)=\{p'\in\reals^k|p^TAp'=0\}\]
		\end{enumerate}
		\item $\left(\D{}{\xi^i}\right)_p$ is also the natural basis of the tangent space. 
		\item 
	\end{enumerate}
	\section{Vector fields}
	\begin{enumerate}
		\item $X_p\in T_p(S)$ a tangent vector
		\item $X:p\mapsto X_p$ a vector field. For example dy
		\begin{enumerate}
			\item $X(x,y)=(x,-y)$ on $\reals^2$
			\item Perpendicular vector fields $X(p)=c\times p,Y(p)=(c\times p)\times p$ on $S^2$ for constant vector $c$
			\item Gradient of $f\in\mcF$
			\item  Coordinate fields formed by natural basis:
			\[\partial_i\defined\left(\D{}{\xi^i}\right):p\mapsto \left(\D{}{\xi^i}\right)_p\cr\tilde{\partial}_j\defined\left(\D{}{\rho^j}\right):p\mapsto \left(\D{}{\rho^j}\right)_p\]
		\end{enumerate}
		\item Motivation: Since for vector field $X$ and every $p$ there are $n$ real numbers $\{X_p^1,\dots,X_p^n\}$ that uniquely determine $X_p=X_p^i(\partial_i)_p$ we can write \[X=X^i\partial_i=\tX^j\tilde{\partial}_j\] 
		\item Component vector field $X$ with respect to $[\xi^i],[\rho^j]$ 
		\[X^i:p\mapsto X^i_p\cr\tX^j:p\mapsto X^j_p\] 
		\item A $C^\infty$  vector field is a vector field whose components are $C^\infty$ with respect to some (therefore all by smooth change of coordinate) coordinate system. 
		\item $\mathcal{T}\defined\{X:X\text{ is a }C^\infty\text{ vector field}\}$ has closure under addition and multiplication by scalar or elements from $\mcF$ i.e. smooth functions on this manifold
	\end{enumerate}
	\section{Tensor fields}
	\begin{enumerate}
		\item A map $F:V_1\times V_2\times\cdots\times V_r\rightarrow W$ where $V_1,\cdots,V_r,W$ are linear spaces is a multilinear map if it is linear in each component. 
		\item Families of multilinearmaps
		\[&[T_p]_r^0\defined
			\{F:\underbrace{T_p\times\cdots\times T_p}_{r \text{ times}}\rightarrow \reals,F\text{ is multilinear}\}\cr
			&[T_p]_r^1\defined\{F:\underbrace{T_p\times\cdots\times T_p}_{r \text{ times}}\rightarrow T_p,F\text{ is multilinear}\}
		\]
		\item $A:p\mapsto A_p\in[T_p]_r^q$ is the tensor field of type $(q,r)$ where $q$ is the contravariant degree and $r$ is the covariant degree. 
		\item $X_1,\cdots,X_r$ are vector fields
		\item $A(X_1,\cdots,X_r):p\mapsto A_p((X_1)_p,\cdots,(X_r)_p)$
		\item $A:(X_1,\cdots,X_r)\mapsto A(X_1,\cdots,X_r)$
		\item $A$ acting on the basis gives components of  $A$ with respect to $[\xi^i]$
		\[&A(\partial_{i_1},\cdots,\partial_{i_r})=A_{i_1\cdots i_r}\cr &A(\partial_{i_1},\cdots,\partial_{i_r})=A_{i_1\cdots i_r}^k\partial_k
		\]
		so that for vector fields $X_1,\cdots,X_r$ \[&A(X_1,\cdots,X_r)=A_{i_1\cdots i_r}X_1^{i_1}\cdots X_r^{i_r}\cr &A(X_1,\cdots,X_r)=A_{i_1\cdots i_r}^kX_1^{i_1}\cdots X_r^{i_r}\partial_k\]
		respectively
		\item Change coordinate to $[\rho^j]$ get
		\[&\tilde{A}_{j_1,\cdots,j_r}=A_{i_1,\cdots,i_r}\left(D{\xi^{i_1}}{\rho^{j_1}}\right)\left(D{\xi^{i_r}}{\rho^{j_r}}\right)&(1.15)\cr
		&\tilde{A}_{j_1,\cdots,j_r}^l=A_{i_1,\cdots,i_r}^k\left(D{\xi^{i_1}}{\rho^{j_1}}\right)\left(D{\xi^{i_r}}{\rho^{j_r}}\right)\left(\D{\rho^l}{\xi^k}\right)&(1.16)\]
	\end{enumerate}
	\section{Submanifolds}
	\begin{enumerate}
		\item $M\subset S$
		\item $n,m$ dimension of $S,M$
		\item Coordinate systems for $S,M$: \[[\xi^i]=[\xi^1,\cdots,\xi^n]\cr[u^a]=[u^1,\cdots,u^n]\]
		\item $M$ is a submnaifold of $S$ if:
		\begin{enumerate}
			\item $\xi^i|_M\in\mcF$ of each $\xi^i$
			\item $B_a^i\defined\left(\D{\xi^i}{u^a}\right)_p$ are linearly independent in $a$
			\item For every $W$ there exists open subset $U$ such that $W=M\cap U$
		\end{enumerate}
		(a) implies embedding $\iota(p)=p$ is $C^\infty$ and (b) implies  $\left(d\iota\right)_p$ is nondegenerate at $p$. For example
		\begin{enumerate}
			\item An open subset of $S$
			\item For $m<n$, define $M\defined\left\{p\in S|\xi^i(p)=c^i,m+1\le i\le n\right\}$ and $u^a\defined\xi^a|M$ for $1\le a\le m$. Note one can also construct supermanifolds with open subsets.
		\end{enumerate}
		\item $\gamma:t\mapsto\gamma(t)$
		\item Using two coordinate systems
		\[&\gamma^a\defined u^a\circ\gamma\cr&\gamma^i\defined\xi^i\circ\gamma\]
		 the tangent vector of $\gamma$ is in both spaces i.e.
		 \[&\left(\der{\gamma^a}{t}\right)_p\left(\partial_a\right)_p\in T_p(M)\cr&\left(\der{\gamma^i}{t}\right)_p\left(\partial_i\right)_p\in T_p(S)\]
		\item By independence condition \[\left(\der{\gamma^i}{t}\right)_p=\left(\D{\xi^i}{u^a}\right)_p\left(\D{\gamma^a}{t}\right)_p\]
		implies one-to-one correspondence between these tangent vectors given by $d\iota$ of embedding $\iota:M\rightarrow S$. So $T_p(M)$ is a subspace of $T_p(S)$. 
		\item Therefore \[\left(\D{}{u^a}\right)_p=\left(\D{\xi^i}{u^a}\right)_p\left(\D{}{\xi^i}\right)_p=B_a^i\partial_i\qquad(1.18)\]
		implies $B_a^i$ is the natural basis vector $\partial_a$ of $M$ with respect to coordinate system $[u^a]$ seen as a vector in $T_p(S)$. 
	\end{enumerate}
	\section{Riemannian metrics}
	\begin{enumerate}
		\item An inner product is a bilinear form that is also symmetric and positive-definite
		\item $\langle,\rangle_p$ an inner product on the tangent space $T_p(S)$
		\item $g_{ij}=\langle\partial_i,\partial_j\rangle$ is in $\mcF$ such that as the tensor field of covariant degree 2 and contravariant degree 1
		\item $[\xi_i][\rho^k]$ coordinate systems
		 \[\partial_i\defined\D{}{\xi^i}\cr\tilde{\partial}_k\defined\D{}{\rho^k}\]
		\item The components of $g$ are determined by $g_{ij}=\langle\partial_i,\partial_j\rangle\in C^\infty$ that maps $g:p\mapsto\langle\left(\partial_i\right)_p,\left(\partial_j\right)_p\rangle_p$ 
		\item For any two tangent vectors \[&D=D^i(\partial_i)_p\cr &D'=D'^i(\partial_i)_p\] in $T_p$ their inner product can be written as $\langle D,D'\rangle_p=g_{ij}(p)D^iD^j$ 
		\item This defines norm fo every tangent vector $D$ as \[\|D\|^2=\langle D,D\rangle_p=g_{ij}(p)D^iD^j\]
		\item $G(p)=[g_{ij}(p)]$ $\tg_{kl}=\langle\tilde{\partial}_k,\tilde{\partial_l}\rangle$
		\[&\tg_{kl}=g_{ij}\left(\D{\xi^i}{\rho^k}\right)\left(\D{\xi^j}{\rho^l}\right)&g_{ij}=\tg_{kl}\left(\D{\rho^k}{\xi^i}\right)\left(\D{\rho^l}{\xi^j}\right)\cr& \tg^{kl}=g^{ij}\left(\D{\rho^k}{\xi^i}\right)\left(\D{\rho^l}{\xi^j}\right)&g^{ij}=\tg^{kl}\left(\D{\xi^i}{\rho^k}\right)\left(\D{\xi^j}{\rho^l}\right)\]
		\item $\tG(p)^{-1}=[\tg^{kl}(p)]$ \[\tg^{kl}=g^{ij}\left(\D{\rho^k}{\xi^i}\right)\left(\D{\rho^l}{\xi^j}\right)\cr g^{ij}=\tg^{kl}\left(\D{\xi^i}{\rho^k}\right)\left(\D{\xi^j}{\rho^l}\right)\]
		\item We also have identity \[g_{ij}g^{jk}=\delta_i^k=\begin{cases}1&k=i\cr0&\ne i\end{cases}\]
		\item For a curve $\gamma:[a,b]\rightarrow S$ the length of the curve is defined by \[\|\gamma\|\defined\int_a^b\left\|\der{\gamma}{t}\right\|dt=\int_a^b\sqrt{g_{ij}\dot\gamma^i\dot\gamma^j}dt\]
		where $\dot\gamma^i$ is the derivative of $\gamma^i\defined\xi^i\circ\gamma$
	\end{enumerate}
	\section{Affine connections}
	\begin{enumerate}
		\item Motivation: For $S$ an open set $\reals^n$, there is natural correspondence between $T_p,T_q$ for $p,q\in S$. However, this may not be the case for arbitrary $S$
		\item 
	\end{enumerate}
	\section{Covariant derivatives}
	\begin{enumerate}
		\item 
	\end{enumerate}
	\section{Flatness}
	\begin{enumerate}
		\item 
	\end{enumerate}
	\section{Autoparallel submanifolds}
	\begin{enumerate}
		\item 
	\end{enumerate}
	\section{Projections of connections}
	\begin{enumerate}
		\item 
	\end{enumerate}
	\section{Embedding curvature}
	\begin{enumerate}
		\item 
	\end{enumerate}
	\section{Riemannian connection}
	\begin{enumerate}
		\item 
	\end{enumerate}
\end{document}